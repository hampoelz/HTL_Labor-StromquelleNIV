\documentclass[a4paper]{hitec}
\settextfraction{0.95}      % reduce left margin

\usepackage{styles/main}
\usepackage{styles/custom}

\labtitle{Stromquelle mit NIV}

\author{Rene Hampölz, Gruppe 6}
\company{HTBLA Weiz, 5BHET}
\date{10. Oktober 2022}

\begin{document}

\maketitletoc
\clearpage

\section{Einführung}

Es soll eine Stromquelle mit einem nicht invertierendem Operationsverstärker (OPV) dimensioniert und aufgebaut werden.
Mit einer Simulation soll die Funktionsweise der Schaltung überprüft werden.

\begin{sagesilent}
    I_a = 0.005
    U_Bat_OPV = 15
    U_O_OPV = 1.8
\end{sagesilent}

Angaben: $I_a = \SIvar{I_a*1000}{\milli\ampere}$, $U_{Bat_{OPV}} = \pm \SIvar{U_Bat_OPV}{\volt}$, $U_g = U_{e0} = R_g \cdot I_a$, OPV: OP27

Datenblatt: $U_{O_{OPV}} \approx \SIvar{U_O_OPV}{\volt}$

\section{Stromquelle mit nicht invertierendem OPV}

\subsection{Schaltung}

\begin{figure}[H]
    \centering
    \begin{circuitikz}
        \draw
          (0,0)    coordinate (in+)
        ++(4,-0.5) coordinate (out+)
          (0,-5)   coordinate (in-)
        ++(4,2)    coordinate (out-)

        (2, -0.5)   node[op amp,yscale=-1]  (opv) {}
        (in+)       to[short,o-]            (opv.+)
        (opv.out)   to[short,-o]            (out+)
        (opv.-)     to[short]               (opv.- |- out-)
                    to[short,-o]            (out-)
                    to[R=$R_g$,-*]          (out- |- in-) coordinate (aux1)
        (out+)      to[short]               ++(1,0) coordinate (aux2)
                    to[R=$R_L$,name=outI,i]     (aux2 |- out-)
                    to[short]               (out-)

        (in-) to[short, o-] ++(2,0) node[ground] {} -- (aux1) -- ++(1,0) coordinate (aux3)

        (in+)   to[open,name=_inV,v] (in-)
        (out+)  to[open,name=outV,v] (out-)
        ($(aux2) + (1,0)$)  to[open,name=_outV,v^] ($(aux3) + (1,0)$)
        ;

        \voltage{_inV}{$U_{e0}$}
        \voltage{_outV}{$U_{a0}$}
        \voltage{outV}{$U_a$}
        \current{outI}{$I_a$}
    \end{circuitikz}
\end{figure}

\subsection{Funktionsweise}
Die Eingangsgröße dieser Quelle ist eine konstante Spannung $U_{e0}$, die Ausgangsgröße ein eingeprägter Strom $I_a$.
Dieser fließt über den Lastwiderstand $R_L$ und den Gegenkopplungswiderstand $R_g$.
Die an $R_g$ abfallende Spannung ist somit $U_g = U_{e0} = R_g \cdot I_a$.
Damit ist mit $R_g$ und $U_{e0}$ der konstante Strom $I_a$ einstellbar.

% \section{Massebezogene Stromquelle}

% \subsection{Schaltung}

% \begin{figure}[H]
%     \centering
%     \begin{circuitikz}
%         \draw
%         (0,-3)    coordinate (in+)
%         ++(0,-2)  coordinate (in-)

%         (0,0)       node[ground] {}
%                     to[R=$R_3$]             ++(3,0) coordinate (aux1)
%         ++(1, -1.5) node[op amp]            (opv) {}
%         (opv.-)     to[short]               (opv.- |- aux1)
%         (aux1)      to[R=$R_4$]             (aux1 -| opv.out) -- (opv.out)
%         (opv.+)     to[short]               (opv.+ |- in+) coordinate (aux2)
%         (in+)       to[R=$R_1$,o-]          (aux2)
%                     to[short]               (aux2 -| opv.out) coordinate (aux3)
%                     to[R=$R_2$]             (opv.out)
%         (aux3)      to[short,-o]            ++(0.5,0) coordinate (out+)
%                     to[R=$R_L$,i>^=$I_a$]   ++(0,-2) coordinate (out-)
%         (in-)       to[short,o-]            ++(1,0)
%                     node[ground] {}         
%                     to[short,-o]            (out-)

%         (in+)   to[open, v=$U_{e0}$] (in-)
%         ($(out+) + (1,0)$)  to[open, v^=$U_a$] ($(out-) + (1,0)$)
            
%         ;
%     \end{circuitikz}
% \end{figure}

% \subsection{Funktionsweise}

\section{Berechnungen}

Da am Gegenkopplungswiderstand $R_g$ die Eingangsspannung $U_{e0}$ abfällt \textit{($U_g = U_{e0}$)}, ergibt sich für die Ausgangsspannung $U_a$ folgende Formel:

\begin{equation*}
    U_a = U_{a0} - U_{e0}
\end{equation*}

Die maximale Ausgangsspannung der Schaltung $U_{a0_{max}}$ wird von der Versorgungsspannung $U_{Bat_{OPV}}$, sowie der Ausgangsspannungsschwankung $U_{O_{OPV}}$ des Operationsverstärkers begrenzt:

\begin{sagesilent}
    U_a0_max = U_Bat_OPV - U_O_OPV
\end{sagesilent}

\begin{align*}
    U_{a0_{max}} &= U_{Bat_{OPV}} - U_{O_{OPV}} \\
    U_{a0_{max}} &= \var{U_Bat_OPV} - \var{U_O_OPV} \\
    U_{a0_{max}} &= \SIvar{U_a0_max}{\volt}
\end{align*}

\smallskip

\textit{(Für minimale Ausgangsspannungsschwankungen können \enquote{rail-to-rail output}-OPVs eingesetzt werden.)}

Um einen möglichst großen maximalen Lastwiederstand $R_{L_{max}}$ bei konstantem Strom $I_a$ zu erzielen, muss die maximale Ausgangsspannung $U_{a_{max}}$ daher möglichst groß gehalten werden.
Somit wird die Eingangsspannung $U_{e0}$ möglichst klein gewählt:

\begin{sagesilent}
    U_e0 = 1
    U_a_max = U_a0_max - U_e0
\end{sagesilent}

\begin{equation*}
    U_{e0} = \SIvar{U_e0}{\volt}
\end{equation*}
\vspace*{-5mm}
\begin{align*}
    U_{a_{max}} &= U_{a0_{max}} - U_{e0} \\
    U_{a_{max}} &= \var{U_a0_max} - \var{U_e0} \\
    U_{a_{max}} &= \SIvar{U_a_max}{\volt}
\end{align*}

Mit diesen Werten kann der maximale Lastwiederstand $R_{L_{max}}$ berechnet werden:

\begin{sagesilent}
    R_L_max = U_a_max/I_a
\end{sagesilent}

\begin{align*}
    R_{L_{max}} &= \frac{U_{a_{max}}}{I_a} \\
    R_{L_{max}} &= \frac{\var{U_a_max}}{\var{I_a}} \\
    R_{L_{max}} &= \SIvar{R_L_max}{\ohm}
\end{align*}

Des Weiteren lässt sich der Gegenkopplungswiderstand $R_g$ berechnen:

\begin{sagesilent}
    R_g = U_e0/I_a
\end{sagesilent}

\begin{align*}
    U_g = U_{e0} &= R_g \cdot I_a \\
    R_g &= \frac{U_{e0}}{I_a} \\
    R_g &= \frac{\var{U_e0}}{\var{I_a}} \\
    R_g &= \SIvar{R_g}{\ohm}
\end{align*}

\bigskip

\section{Simulation}

\begin{sagesilent}
    simulation_data = LTSpice.plot_data("src/01 Stromquelle mit NIV.txt")
    simulation_data_Iout = simulation_data["I(Rl)"]

    simulation_plot = list_plot(
        simulation_data_Iout,
        axes_labels=['$R_L$ in $\Omega$', '$I_a$ in A'],
        figsize=[5.5,3],
        plotjoined=True,
        frame=True,
        color="red"
    )
\end{sagesilent}

\begin{figure}[H]
    \centering
    \sageplot{simulation_plot}
    \caption{Ausgangskennlinie \textbf{$I_a = f(R_L)$} der Simulation}
\end{figure}

\section{Auswertung}

\begin{sagesilent}
    measures = Measures(
        [
            '$R_L$ in $\Omega$',
            '$I_a$ in mA', 
            '$U_a$ in V'
        ], [
            ['0', '5', '0.005'],
            ['300', '5', '1.4'],
            ['600', '5', '2.8'],
            ['900', '5', '4.3'],
            ['1200', '5', '5.7'],
            ['1500', '5', '7.2'],
            ['1800', '5', '8.9'],
            ['2100', '5', '10.8'],
            ['2400', '5', '11.8'],
            ['2700', '4', '12.7'],
            ['3000', '4', '13.1'],
            ['3300', '3', '13.2']
    ])

    index_RL = 0
    index_Ia = 1
    index_Ua = 2
\end{sagesilent}

\subsection{Messdaten}

\begin{center}
    \renewcommand{\arraystretch}{1.2}
    \var{measures.table}
\end{center}

\subsection{Grafische Darstellung}

\begin{sagesilent}
    f_RL = list_plot(
        measures.plot_data(index_RL, index_Ia),
        axes_labels=['$R_L$ in $\Omega$', '$I_a$ in mA'],
        figsize=[5.5,2],
        color='purple'
    )

    f_RL += line(
        measures.g1d_smooth(index_RL, index_Ia, sigma=0.75, connect=True),
        linestyle="--",
        color='red'
    )

    f_Ua = list_plot(
        measures.plot_data(index_Ua, index_Ia),
        axes_labels=['$U_a$ in V', '$I_a$ in mA'],
        figsize=[5.5,2],
        color='purple'
    )

    f_Ua += line(
        measures.g1d_smooth(index_Ua, index_Ia, sigma=0.95, connect=True),
        linestyle="--",
        color='blue'
    )
\end{sagesilent}

\begin{figure}[H]
    \centering
    \sageplot{f_RL}
    \caption{Ausgangskennlinie \textbf{$I_a = f(R_L)$} mit gemessene Werte}
\end{figure}

\begin{figure}[H]
    \centering
    \sageplot{f_Ua}
    \caption{Ausgangskennlinie \textbf{$I_a = f(U_a)$} mit gemessene Werte}
\end{figure}

\clearpage

\subsection{Bemerkung}

Die Schaltung liefert bis zum ermittelten maximalen Lastwiederstand $R_{L_{max}}$ einen konstanten Ausgangsstrom $I_a$.

\subsection{Verwendete Komponenten}

\medskip

\begin{devicelist}
    \devicecat{Widerstands-Dekade}{2}{%
        \device{ET-MTL1-RD23}{R_g}
        \device{ET-MTL1-RD29}{R_L}
    }
    \devicecat{Spannungsquelle}{1}{%
        \device{ET-MTL1-NG03}{U_{e0}}
    }
    \devicecat{Multimeter}{2}{%
        \device{ET-MTL1-DM20}{I_a}
        \device{ET-MTL1-DM22}{U_a}
    }
\end{devicelist}

\vfill

\IncludeHistoryTimeline

\end{document}